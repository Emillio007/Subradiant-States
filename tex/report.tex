\documentclass{article}
\usepackage[utf8]{inputenc}
\usepackage{graphicx}
\usepackage[margin=1.0in]{geometry}
\usepackage{float}
\usepackage{wrapfig}
\usepackage{textcomp}
\usepackage{mathtools, nccmath}
\usepackage{amssymb}
\usepackage{amsmath}
\usepackage{mathrsfs}
\usepackage{dsfont}
\usepackage{lmodern}
\usepackage{textcomp}
\usepackage{hyperref}
\usepackage{physics}
\usepackage{listings}
%\setlength{\parindent}{0pt}
\usepackage{setspace}
\usepackage[sorting=none]{biblatex}
\usepackage{caption}
\usepackage{subcaption}
\usepackage[eng]{nbi}
%\addbibresource{lit.bib}
\onehalfspacing

\supervisor{Klaus Mølmer}
\project{Project outside course scope}
\author{Emil H. Henningsen}
\title{Investigation of Collective Atomic Excitation States}
\subtitle{}
\institute{Niels Bohr Institute}
\department{Hy-Q - Center for Hybrid Quantum Networks}
\email{tzs820@alumni.ku.dk}
\handindate{$14^{th}$ of June, 2024}
\defencedate{$21^{st}$ of June, 2024}

\begin{document}

\maketitle

\section*{Abstract}

TODO. 

\section*{Acknowledgements}

\newpage
\tableofcontents

\section{Introduction}

The collective excitation states of an atomic array is a well-known phenomenon in the physics of light-matter interaction (REF). In the field of quantum information processing, long-lived quantum states are desired for implementing both quantum communication protocols and quantum computing. Certain excited states of atomic arrays display greatly amplified (superradiant) or suppresed (subradiant) decay rates, when compared to a single atom in vacuum. Such states arise in the interplay of atoms arranged closely together and the surrounding electromagnetic field. In the following, exciting new geometries and variations are investigated within the scope of finding highly subradiant states. The model used for these calculations is a second quantized approach, in which the atoms are assumed dipoles, and the suppresed radiance is manifested as destructive interference between these (???). Furthermore, theoretical considerations are presented to elaborate on expectations and results of these calculations. This project is restricted to radiation in free-space.

\section{Theory}

The model is developed from a classical perspective, but will in the end be used to describe the quantized field and therefore be a fully quantum model (REF Gruner og Welsch). The model is used to describe the interaction between atoms and the empty-space electromagnetic field. The model is capable of describing both homogenous and inhomogenous Kramers-Kronig dielectrics (REF Gruner og Welsch), which makes it desirable for in-depth studies of collectively excited atomic arrays in various settings and environments. Due to the quantities of interest in this project, the framework is restricted to free space in the following. 

List of assumptions and approximations (REF Asenjo):
\begin{itemize}\label{list:assumptions}
    \item All photons are mediated in vacuum, i.e. $\varepsilon(\bar{r},\omega) = \varepsilon_0$,
    \item Targeting a single transition in the atoms, the dipole-interaction between atoms and the field happens at a narrow band-width, i.e. $\omega \rightarrow \omega_0$,
    \item The atoms are tightly trapped, such that their positions can be treated as stationary points. 
    \item Interatomic distances are smaller than the wavelength of emitted photons, $d < \lambda_0$. With this, the retardation of the field between the atoms can be neglected, 
    \item There are no strongly-coupled modes between the field and the atoms (which would e.g. be the case, if the atoms were placed in an optical cavity). Together with the abovementioned assumption, the emission is in the Markovian regime, which means "the system will have no memory of its past" (REF Deutsch) once the photon has escaped. 
\end{itemize}
The latter two points allow for the promotion of field and dipole moment to quantum operators, after which the field described by Green's tensor and radiation-scattering dipoles can be coherently described by atomic coherence operators. In particular, the Green's tensor under these assumptions become (REF Asenjo, equation 6):

\begin{equation}\label{eq:Greens}
    \underline{G}_0 (\bar{r}_{ij}, \omega_0) = \frac{e^{ik_0r}}{4\pi k_0^2 r^3} \cdot \left[ (k_0^2r^2 + ik_0r - 1) \mathds{1}_3 + (-k_0^2 r^2 -3ik_0r + 3) \frac{\ket{\bar{r}_{ij}}\bra{\bar{r}_{ij}}}{r^2} \right].
\end{equation}
Where $r$ is the norm of the displacement between atoms i and j: $\bar{r}_{ij}$, $k_0=\frac{\omega_0}{c}$ is the norm of the wave vector, $\mathds{1}_3$ is the three-dimensional identity matrix and $\ket{\bar{r}_{ij}}\bra{\bar{r}_{ij}}$ is the outer product of the displacement vector. 

The effective Hamiltonian (REF Asenjo), which is subject to diagonalization:
\begin{equation}\label{eq:Heff}
    \begin{split}
        \hat{H}_{eff} &= - \mu_0 \cdot \omega_0^2 \cdot \sum_{i,j = 1}^N \bar{\mathscr{D}}^\dagger G_0(\bar{r_i}, \bar{r_j}, \omega_0) \bar{\mathscr{D}} \sigma_+^i \sigma_-^j, \\
        &=- \mu_0 \cdot \omega_0^2 \cdot \|\bar{\mathscr{D}}\|^2 \cdot \sum_{i,j = 1}^N \hat{n}^T G_0(\bar{r_i}, \bar{r_j}, \omega_0) \hat{n} \sigma_+^i \sigma_-^j. \\
    \end{split}
\end{equation}
Where $\mu_0$ is the vacuum permeability, $\omega_0$ is the atomic transition frequency, $\bar{\mathscr{D}}$ is the dipole vector at atom j (however, for most applications, equal directioned dipoles are chosen) and $\sigma_\pm^i$ is the i'th atom's coherence operators. In the last line, the dipole amplitude is separated from the direction, which proves useful when converting to a dimensionless model \ref{sec:dimless}.

TODO: Med udgangspunkt i Asenjo-Garcia opsummér de gjorte antagelser og de deraf følgende resultater for Greens funktion. Hvorfor kan vi tillade os at diagonalisere ikke-hermitisk Hamilton og forvente retvisende egenværdier? Se på fx sum af sandsynlighed osv. Trace af hermitisk matrix er bevaret. 

Kæde: Foton kan ikke undslippe ortogonalt til kæden. Hvorfor? -> "Spin-bølge" tilstand i gitteret. -> faststoffysik-sprog. Subradiante tilstande for bølgevektorer indenfor første Brillouin-zone. 

Greens tensor divergerer for i=j tilfælde. Diagonalen indsættes ved håndkraft. 

Forudsigelser for N=2 tilfælde? Problemer! Sammenligning med Adrian N=3 kæde. Vidde på egenværdier gør det svært for algoritmen at være præcis, derfor skal der ikke så meget til, før vi havner på den forkerte side.  DISKUSSION?

\subsection{}

\subsection{Time-evolution of excitation}

Er dette interessant? Tidslig udvikling kan opnås ved invertering, se noter møde 4/4.

\section{Method}

Issue: Full size of Hilbert-space scales exponentially -> high memory requirements.

\subsection{Block hamiltonian}

Excitation number conserved -> block-diagonal matrix (evt. inkluder N=2 tilfælde fra noter) -> NxN matrix problem (N(N+1)/2 x N(N+1)/2 for 2-excitation states ...)

\subsection{Dimensionless computation}\label{sec:dimless}

\section{Results \& Discussion}

Forudsigelser for N=2 tilfælde? Problemer! Sammenligning med Adrian N=3 kæde. Vidde på egenværdier gør det svært for algoritmen at være præcis, derfor skal der ikke så meget til, før vi havner på den forkerte side. TEORI?

TODO: Evt. put noget om, at det er svært at addressere disse subradiante tilstande, og at der indtil videre ikke findes nogle gode protokoller for netop dette (kilde? Udover introduktion in Asenjo-Garcia et al.). 

\subsection{Linear chain}

\subsubsection{Varying polarization in linear chain}

\subsubsection{Broken chain}

\subsection{Circular chain}

\subsection{???}

\subsection{Further research}

Other geometries of interest? Helix, . How is decay affected by curvature?
Any specific geometries that make efficient and high-fidelity communication protocols available?

Multiexcitation states and their behaviour (REF, Asenjo-Garcia, Israeli guy)

The framework developed by Grüner \& Welsch (REF) can be extended to include propagation in other media and guided modes. This opens up for other interesting directions to go. E.g. how circular lattices around a fibre with a guided mode acts. How does it affect the decay rates of the system? 

\section{Conclusion}

\newpage
TODO: bibliography

\end{document}