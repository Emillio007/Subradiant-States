\documentclass{article}
\usepackage[utf8]{inputenc}
\usepackage{graphicx}
\usepackage[margin=1.0in]{geometry}
\usepackage{float}
\usepackage{wrapfig}
\usepackage{textcomp}
\usepackage{mathtools, nccmath}
\usepackage{amssymb}
\usepackage{amsmath}
\usepackage{lmodern}
\usepackage{textcomp}
\usepackage{hyperref}
\usepackage{physics}
\usepackage{listings}
%\setlength{\parindent}{0pt}
\usepackage{setspace}
\usepackage[sorting=none]{biblatex}
\usepackage{caption}
\usepackage{subcaption}

\begin{document}

\title{Wigner-Weisskopf for to atomer og fri vakuum}
\author{Emil Henningsen}
\maketitle

Følger Deutsch' udledning. \footnote[1]{Deutsch - Spontaneous Emission: Wigner-Weisskopf Theory}

\begin{equation}
    \hat{H} = 2\pi \gamma_{NV} S_z B_z = \pi \hbar \gamma_{NV} B_z \begin{pmatrix} 1 & 0 \\ 0 & -1 \\ \end{pmatrix}
\end{equation}

Eigenstates: ${\ket{0}, \ket{1}}$

\begin{equation}
    \hat{H} = 2 \pi \gamma_{NV} (S_z B_z + S_x B_x) = \pi \hbar \gamma_{NV} \left[ B_z \begin{pmatrix} 1 & 0 \\ 0 & -1 \\ \end{pmatrix} + B_x \begin{pmatrix} 0 & 1 \\ 1 & 0 \\ \end{pmatrix} \right]
\end{equation}

\end{document}