\documentclass{article}
\usepackage[utf8]{inputenc}
\usepackage{graphicx}
\usepackage[margin=1.0in]{geometry}
\usepackage{float}
\usepackage{wrapfig}
\usepackage{textcomp}
\usepackage{mathtools, nccmath}
\usepackage{amssymb}
\usepackage{amsmath}
\usepackage{lmodern}
\usepackage{textcomp}
\usepackage{hyperref}
\usepackage{physics}
\usepackage{listings}
%\setlength{\parindent}{0pt}
\usepackage{setspace}
\usepackage[sorting=none]{biblatex}
\usepackage{caption}
\usepackage{subcaption}
\usepackage[eng]{nbi}
%\addbibresource{lit.bib}
\onehalfspacing

\supervisor{Klaus Mølmer}
\project{Project outside course scope}
\author{Emil H. Henningsen}
\title{Investigation of Collective Atomic States}
\subtitle{}
\institute{Niels Bohr Institute}
\department{Hy-Q - Center for Hybrid Quantum Networks}
\email{tzs820@alumni.ku.dk}
\handindate{$14^{th}$ of June, 2024}
\defencedate{$21^{st}$ of June, 2024}

\begin{document}

\maketitle

\section*{Abstract}

TODO. 

\section*{Acknowledgements}

\newpage
\tableofcontents

\section{Introduction}

The collective excitation states of an atomic array is a well-known phenomenon in the physics of light-matter interaction. In the field of quantum information processing, long-lived quantum states are desired for implementing both quantum communication protocols and quantum computing. Certain excited states of atomic arrays display greatly amplified (superradiant) or suppresed (subradiant) decay rates, when compared to a single atom in vacuum. Such states arise in the interplay of atoms arranged closely together and the surrounding electromagnetic field. In the following, exciting new geometries and variations are investigated within the scope of finding highly subradiant states. The model used for these calculations is a fully quantum approach, in which the atoms are assumed dipoles, and the suppresed radiance is manifested as destructive interference between these (???). Furthermore, theoretical considerations are presented to elaborate on expectations and results of these calculations. 

\section{Theory}

TODO: Med udgangspunkt i Asenjo-Garcia opsummér de gjorte antagelser og de deraf følgende resultater for Greens funktion. 

\section{Method}

\section{Results \& Discussion}

TODO: Evt. put noget om, at det er svært at addressere disse subradiante tilstande, og at der indtil videre ikke findes nogle gode protokoller for netop dette (kilde? Udover introduktion in Asenjo-Garcia et al.). 

\section{Conclusion}

\newpage
TODO: bibliography

\end{document}